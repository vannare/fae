% incl/pre/conf.tex : generelle indstillinger
% ------------------------------------------------------------------------------
% Denne fil indeholder indstillinger for opsætningen af rapporten


% Brugerdefinerede farver ------------------------------------------------------
% Officielle AAU-farver
% https://www.design.aau.dk/om-aau-design/Farver/
\definecolor{AAUblue1}{RGB}{ 33, 26, 82}
\definecolor{AAUblue2}{RGB}{ 89, 79,191}
\definecolor{AAUgrey1}{RGB}{ 84, 97,110}
\definecolor{AAUgrey2}{RGB}{104,119,132}
\definecolor{AAUgrey3}{RGB}{162,172,182}
\definecolor{AAUgrey4}{RGB}{222,223,226}
\definecolor{AAUgreen}{RGB}{157,187, 29}
\definecolor{AAUteal} {RGB}{ 94,150,149}
\definecolor{AAUred}  {RGB}{223,103, 82}

% Sidehoved og sidefod ---------------------------------------------------------
% http://mirrors.dotsrc.org/ctan/macros/latex/contrib/fancyhdr/fancyhdr.pdf
\fancyhead{}                            % ryd standardindhold
\fancyhead[LE,RO]{Group \projectgroup}  % venstre på lige sider, højre på ulige
\fancyhead[RE]{\leftmark}               % højre på ulige sider
\fancyhead[LO]{\rightmark}              % venstre på ulige sider
\fancyfoot{}                            % ryd standardindhold
\fancyfoot[CE,CO]{\thepage}             % centrerede sidetal
\pagestyle{fancy}                       % aktiver fancy stil

% Overskrifter -----------------------------------------------------------------
% http://mirrors.dotsrc.org/ctan/macros/latex/contrib/titlesec/titlesec.pdf

% Kapitler vises med et tal i en farvet kasse
\titleformat
{\chapter}
[hang]
{\Huge\bfseries\color{white}}
{\colorbox{AAUblue1}{\makebox[30pt]{\strut\thechapter}}}
{0.5em}
{\Huge\bfseries\color{AAUblue1}}

% Dele skrives på to linjer
\titleformat
{\part}
[display]
{\Huge\bfseries\color{AAUblue1}\filcenter}
{\huge\partname{} \thepart}
{0em}
{}

% Juster titel og indholdsfortegnelsen mht. appendiks-siden
\renewcommand{\appendixname}{Appendix}
\renewcommand{\appendixtocname}{Appendicer}
\renewcommand{\appendixpagename}{\textcolor{AAUblue1}{Appendicer}}
\xpatchcmd{\addappheadtotoc}{{chapter}}{{part}}{}{}

% Figurindstillinger -----------------------------------------------------------
% Indstillinger for figurtekst
% http://mirrors.dotsrc.org/ctan/macros/latex/contrib/caption/caption-eng.pdf
\captionsetup{font=it,labelfont=bf}

% Nyttige TiKZ-udvidelser
% https://tex.stackexchange.com/q/42611
\usetikzlibrary{arrows.meta}  % forskellige pilespidser
\usetikzlibrary{calc}         % udvidede beregninger med koordinater

% Prædefinerede styles til punkter og kanter i en graf
\tikzstyle{point} = [fill,shape=circle,minimum size=3pt,inner sep=0pt]
\tikzstyle{edge}  = [fill=white,midway,inner sep=1pt]

% Algoritmer -------------------------------------------------------------------
% Indstillinger for algorithm-blokke
\floatname{algorithm}{Algoritme}
\renewcommand{\listalgorithmname}{Algoritmer}

% Kildekode --------------------------------------------------------------------
% Syntax highlighting for Python-kode
% https://en.wikibooks.org/wiki/LaTeX/Source_Code_Listings
\lstdefinestyle{custompy}{
  belowcaptionskip=\baselineskip,
  breaklines=true,
  language=Python,
  showstringspaces=false,
  basicstyle=\footnotesize\ttfamily,
  keywordstyle=\bfseries\color{AAUblue1},
  commentstyle=\itshape\color{AAUgrey3},
  identifierstyle=\color{AAUblue2},
  stringstyle=\color{AAUteal},
}
\lstset{language=Python,style=custompy,captionpos=b}
\renewcommand{\lstlistingname}{Kildekode}
\renewcommand{\lstlistlistingname}{Kildekode}

% Indstillinger for autoref ----------------------------------------------------
% https://en.wikibooks.org/wiki/LaTeX/Labels_and_Cross-referencing#autoref
\def\partautorefname{Del}
\def\chapterautorefname{Kapitel}
\def\sectionautorefname{Afsnit}
\def\subsectionautorefname{Underafsnit}
\def\figureautorefname{Figur}
\def\tableautorefname{Tabel}
\def\algorithmautorefname{Algoritme}
\def\lstinputlistingautorefname{Kildekode}
